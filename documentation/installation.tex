\section{Building and Installing ASAGI}

\subsection{Pre-requirements}

\subsubsection{MPI}

ASAGI makes use of the RMA (Remote Memory Access) API of the MPI-2 standard to transfer data. An MPI library that supports the new standard is required.

\subsubsection{NetCDF}

ASAGI uses the NetCDF library\footnote{\url{http://www.unidata.ucar.edu/software/netcdf/}} to load data files. Since most standard NetCDF packages do not support the new NetCDF-4 C++ interface, you may need to compile the NetCDF library yourself.

To enable the NetCDF-4 C++ support, make sure to add the option ``enable-cxx-4'' to the Configure script:

\lstset{language=bash}
\begin{lstlisting}
./configure --enable-shared --enable-cxx-4 --prefix=<install_dir>
make
make install
\end{lstlisting}

\subsection{Compilation}

To generate the Makefiles, CMake is used. For CMake it is recommend to keep source and build directory apart:

\lstset{language=bash}
\begin{lstlisting}
mkdir build
cd build
cmake <path/to/asagi_sources>
\end{lstlisting}

Several environment variables affect the behavior of CMake. They must be set before running ``cmake''.

\begin{description}
 \item[Compiler] The compiler can be selected by setting \texttt{CC} (C compiler), \texttt{CXX} (C++ compiler) and \texttt{FC} (FORTRAN compiler) environment variables. C and FORTRAN compiler are only required for C and FORTRAN examples.
 \item[Libraries] The \texttt{CMAKE\_PREFIX\_PATH} is used when searching for the NetCDF and PNG library. If NetCDF was configured with \texttt{--prefix=<install\_dir>} for example, set \texttt{CMAKE\_PREFIX\_PATH=<install\_dir>}.
\end{description}

Besides the environment variables, you can change the behavior by setting internal CMake variables. They can be configured by adding one ore more \texttt{-D<variable>=<value>} options when running ``cmake''. These variables can also be changed later with the following command:
\lstset{language=bash}
\begin{lstlisting}
ccmake <path/to/asagi_build>
\end{lstlisting}

The important variables are listed below. Most of the variables are ASAGI specific and will not work with other CMake projects.
\begin{description}
 \item[CMAKE\_BUILD\_TYPE = Debug\textbar Release] When set to ``Debug'', addi\-tion\-al run-time checks are enabled as well as debug messages. \texttt{[Release]}
 \item[CMAKE\_INSTALL\_PREFIX] Installation directory for ASAGI.\\ \texttt{[/usr/local/]}
 \item[EXAMPLES = ON\textbar OFF] Compile example programs. \texttt{[OFF]}
 \item[FORTRAN\_SUPPORT = ON\textbar OFF] Compile with FORTRAN support.\\ \texttt{[ON]}
 \item[STATIC\_LIB = ON\textbar OFF] Compile ASAGI as a static library. \texttt{[OFF]}
 \item[TESTS = ON\textbar OFF] Compile tests. \texttt{[OFF]}
 \item[THREADSAFETY = ON\textbar OFF] If enabled all ASAGI functions are thread-safe. This is required, for example, if ASAGI is used in hybrid MPI/""OpenMP programs. \texttt{[ON]}
\end{description}

\subsection{Installation}

Installation is currently not supported.